\documentclass{ULBreport} % Llama a tu plantilla adaptada

% Configuración de enlaces (esto sí se queda en el main o se puede mover al .cls)
\hypersetup{
    colorlinks=true,
    linkcolor=black,
    filecolor=magenta,      
    urlcolor=blue,
    citecolor=black,
}

\begin{document}

% 1. GENERAR LA PORTADA CON EL COMANDO ESTRUCTURADO
\titleUCLM{
    title={Valorant Performance Evaluation \\ Un sistema de Inteligencia Artificial para la evaluación objetiva del rendimiento individual en competiciones de esports},
    course={Desarrollo e Integración de Servicios de IA},
    author={{\large Nicolás Celis Moreno}\\[0.7em]
            {\large Erick Mesias Rosero Lesano}\\[0.7em]
            {\large Johann Ismael Ordoñez Echeverría}\\[0.7em]},
    teacher={Julio Alberto López},
    date={2026},
    logoL={Logo_UCLM_40.jpg},
    logoR={esi.png},
    watermark={uclm_fondo.png}
}

% 2. ÍNDICES (Se generan con cuadro gris automáticamente al usar la clase report)
\tableofcontents
\listoftables
\listoffigures


% 3. CONTENIDO PRINCIPAL
% -----------------------------------------------------------------
% NOTA: Usamos \chapter para que el paquete fncychap dibuje el cuadro gris
\chapter{Definición del problema y determinación del alcance}

En este capítulo se detalla la definición del problema seleccionado para la evaluación del proyecto de la asignatura Desarrollo e Integración de Servicios de IA, así como la determinación de su alcance.

\section{Contexto y definición del problema}

En el ecosistema competitivo de Valorant, los equipos que participan en las Valorant Challengers Leagues (VCL) representan el nivel Tier 2 del circuito profesional, funcionando como segunda división o cantera del sistema VCT. Estas organizaciones operan en un entorno altamente competitivo, con presupuestos limitados y márgenes estratégicos reducidos, donde cada decisión relacionada con fichajes, renovaciones o promociones internas tiene implicaciones deportivas, organizativas y económicas significativas.

Más allá del impacto financiero, uno de los principales desafíos que enfrentan estas organizaciones es la evaluación objetiva del rendimiento individual en un juego cuya naturaleza es profundamente contextual y multidimensional. A diferencia de métricas tradicionales basadas únicamente en resultados visibles como asesinatos, daño total o puntuaciones agregadas el desempeño en Valorant depende de variables que interactúan de forma compleja: el rol desempeñado (Duelist, Controller, Sentinel o Initiator), el agente seleccionado, el mapa disputado, la economía del equipo, la gestión de utilidades y la influencia en rondas críticas.

En la práctica, los procesos actuales de análisis en equipos Tier 2 combinan revisión manual de partidas (VOD review), interpretación subjetiva del cuerpo técnico y consulta de estadísticas públicas como K/D, ACS o ADR. Aunque estos métodos aportan información relevante, presentan limitaciones estructurales. El análisis manual es costoso en tiempo y depende de la experiencia del analista; las métricas agregadas tienden a favorecer perfiles orientados al enfrentamiento directo (fraggers o duelists); y la comparación entre jugadores de roles distintos suele realizarse sin una normalización contextual adecuada.

Esta situación genera un margen de error estimado entre el 30\% y el 45\% en decisiones relacionadas con evaluación individual, debido a sesgos cognitivos que priorizan acciones visibles y subestiman contribuciones menos evidentes, como el uso eficiente de utilidades, la creación de espacio, la presión estratégica o la consistencia táctica dentro del rol asignado. En consecuencia, las organizaciones pueden sobrevalorar determinados perfiles y subestimar otros cuya aportación es crítica para la estructura colectiva del equipo.

El problema que se aborda en este proyecto no se limita a generar un ranking general de jugadores, sino a diseñar un sistema de inteligencia artificial capaz de evaluar el desempeño individual dentro del rol específico que ocupa cada jugador, considerando el contexto competitivo en el que se produce dicho rendimiento. El sistema debe aprender la importancia relativa de múltiples métricas técnicas y estratégicas, integrando información procedente de distintas fuentes estructuradas, con el objetivo de producir rankings diferenciados por rol que sean coherentes, explicables y alineados con el análisis experto.

Desde un punto de vista formal, el problema se plantea como un modelo de optimización orientado a ranking, donde cada jugador es representado por un conjunto de variables cuantitativas contextualizadas por agente, mapa y arma. El sistema debe determinar una combinación ponderada de estas variables que permita ordenar a los jugadores de forma consistente, reduciendo el sesgo hacia métricas superficiales y proporcionando una herramienta de apoyo a la decisión para organizaciones Tier 2.

El sistema no pretende reemplazar el criterio humano, sino estructurarlo, reducir su variabilidad y complementar el análisis tradicional con un enfoque reproducible y formalizado. En este sentido, el proyecto aborda simultáneamente un problema competitivo de mejorar la calidad del scouting y la evaluación interna y un problema metodológico para formalizar la evaluación del rendimiento en esports como un sistema interpretable y transferible a otros entornos competitivos digitales.

% FIGURA 1.1
\begin{figure}[H]
    \centering
    \includegraphics[width=0.9\textwidth]{figura1.png}
    \caption{Arquitectura general del sistema VPER: integración de múltiples fuentes de datos y generación de rankings por rol.}
    \label{fig:arquitectura}
\end{figure}

\section{Objetivos generales y específicos}

El problema que se aborda en este proyecto puede formularse de la siguiente manera:

\textit{¿Cómo diseñar un sistema de inteligencia artificial capaz de evaluar de forma objetiva, consistente y explicable el rendimiento individual de jugadores en un torneo competitivo de Valorant, generando rankings diferenciados por rol?}

Para dar respuesta a esta cuestión, se debe: Diseñar, implementar y validar un sistema de inteligencia artificial capaz de evaluar de forma objetiva, consistente y explicable el rendimiento individual de jugadores en competiciones VCL, generando rankings diferenciados por rol que sirvan como apoyo a decisiones de scouting en equipos Tier 2.

Junto con los objetivos especificos que son los siguientes:

\begin{itemize}
    \item \textbf{Formalizar} el concepto de rendimiento óptimo por rol en Valorant competitivo.
    \item \textbf{Integrar y estructurar} datos históricos procedentes de múltiples APIs públicas.
    \item \textbf{Diseñar} un modelo de optimización orientado a ranking que aprenda pesos contextuales por agente, mapa y arma.
    \item \textbf{Reducir} el sesgo estructural hacia \textit{duelists} y \textit{fraggers}.
    \item \textbf{Evaluar} la coherencia del ranking con valoraciones expertas del staff técnico.
    \item \textbf{Diseñar} el sistema como producto integrable en procesos internos de análisis competitivo.
\end{itemize}

% FIGURA 1.2
\begin{figure}[H]
    \centering
    \includegraphics[width=0.8\textwidth]{figura2.png}
    \caption{Objetivos del sistema VPER para la evaluación objetiva del rendimiento en esports.}
    \label{fig:objetivos}
\end{figure}

\section{Determinación del alcance}

El sistema VPER tiene como objetivo delimitar con precisión qué funcionalidades forman parte del proyecto y cuáles quedan explícitamente excluidas, garantizando una ejecución realista dentro del marco temporal de la asignatura y coherente con el desarrollo de un sistema de inteligencia artificial concebido como producto.

El sistema se centrará exclusivamente en la evaluación objetiva del rendimiento individual de jugadores que compiten en torneos pertenecientes a las Valorant Challengers Leagues (VCL), generando rankings diferenciados por rol competitivo (Duelist, Controller, Sentinel e Initiator). Esta delimitación responde a la necesidad de mantener coherencia estructural en la comparación de jugadores, evitando comparaciones transversales entre funciones estratégicamente distintas dentro del juego.

El alcance funcional del proyecto comprende, en primer lugar, la integración de datos históricos estructurados procedentes de múltiples fuentes públicas (Riot API, VLR.gg, Tracker.gg y HenrikDev API). Esta integración no se limita a la mera recopilación de estadísticas agregadas, sino que implica la consolidación, normalización y contextualización de métricas relevantes, tales como rendimiento por agente, impacto por mapa y eficiencia asociada al uso de armas. El objetivo es construir un dataset consistente que permita modelar el desempeño individual desde una perspectiva multidimensional y contextualizada.

En segundo lugar, el sistema incluirá el diseño de un modelo de optimización orientado a ranking, cuya función consistirá en aprender los pesos asociados a cada métrica de rendimiento. Este modelo permitirá generar ordenamientos diferenciados por rol, reduciendo el sesgo estructural hacia métricas centradas exclusivamente en daño o asesinatos, y proporcionando una evaluación más equilibrada del impacto competitivo.

Asimismo, el alcance contempla la evaluación técnica del modelo mediante métricas cuantitativas y la validación cualitativa de los rankings generados a través de su alineación con valoraciones expertas del staff técnico o analistas del dominio competitivo. Esta fase resulta esencial para garantizar que el sistema no se limite a optimizar indicadores estadísticos, sino que produzca resultados interpretables y coherentes con la realidad estratégica del juego.

\subsection{Planificación}

La duración estimada del proyecto es de doce semanas, distribuidas en fases que siguen el ciclo de vida de un proyecto de Machine Learning.

En una primera fase se realizará la definición conceptual del problema, incluyendo la formalización matemática del modelo de ranking y la definición de métricas de evaluación alineadas con el dominio competitivo. Posteriormente, se desarrollará la fase de integración de datos, en la que se implementarán los mecanismos de extracción y consolidación de información procedente de las distintas APIs seleccionadas.

La tercera fase consistirá en el diseño y entrenamiento del modelo de optimización, donde se ajustarán los pesos asociados a cada métrica de rendimiento. A continuación, se llevará a cabo la evaluación técnica del sistema, analizando la coherencia de los rankings generados y su alineación con valoraciones expertas. Finalmente, se desarrollará un prototipo experimental de visualización de resultados que permita demostrar la aplicabilidad práctica del sistema.

Esta planificación garantiza que el proyecto no se limite al entrenamiento de un modelo aislado, sino que abarque todas las etapas necesarias para concebir un sistema de IA como producto funcional.

\subsection{Recursos Humanos: asignación de tareas y capacidad}

El desarrollo del sistema VPER requiere la colaboración de perfiles complementarios que reflejan una estructura realista de proyecto de inteligencia artificial.

El ML Engineer o AI Researcher será responsable de la formulación del problema como modelo de optimización, del diseño de la función objetivo y del entrenamiento del sistema. Este perfil se encargará también del análisis de explicabilidad e interpretación de los pesos aprendidos.

El Data Engineer o Data Analyst asumirá la responsabilidad de integrar las distintas fuentes de datos, desarrollar los procesos de limpieza, normalización y estructuración, y garantizar la consistencia del dataset final. Dado que el sistema depende de múltiples APIs y fuentes externas, este rol resulta crítico para asegurar la calidad del modelo.

El Experto en dominio, en este caso un analista competitivo de Valorant, desempeñará un papel esencial en la validación cualitativa de los rankings generados. Su función será evaluar si los resultados producidos por el sistema son coherentes con la realidad estratégica del juego competitivo.

El stakeholder principal del proyecto será una organización competitiva, academia o equipo semiprofesional de Valorant interesado en mejorar sus procesos de scouting y evaluación de jugadores. El sistema se concibe como herramienta de apoyo a la decisión y no como sustituto del juicio experto.

\subsection{Viabilidad. Estudio del ROI}

La viabilidad económica del sistema VPER se analiza considerando un escenario realista dentro del ecosistema competitivo de Valorant Tier 2 (VCL), donde los equipos operan con presupuestos anuales estimados entre 45.000 y 250.000 dólares, y los salarios promedio por jugador oscilan entre 1.200\,€ y 3.500\,€ mensuales, pudiendo alcanzar los 5.000\,€ en organizaciones con mayor respaldo financiero.

El desarrollo del sistema se estima bajo los siguientes supuestos:

\begin{itemize}
    \item Duración del proyecto: 12 semanas
    \item Equipo de trabajo: 3 personas
    \item Dedicación media: 10 horas semanales por persona
    \item Coste medio estimado por hora: 45\,€
\end{itemize}

Las horas totales invertidas serían:

\[
H = 3 \times 10 \times 12 = 360 \text{ horas}
\]

El coste total estimado del desarrollo sería:

\[
C = 360 \times 45 = 16.200\,€
\]

En el contexto Tier 2, un contrato anual de un jugador puede situarse entre 14.400\,€ y 42.000\,€, dependiendo del nivel competitivo y la región. Un fichaje erróneo no solo implica el coste salarial directo, sino también costes indirectos asociados a bajo rendimiento competitivo, pérdida de oportunidades de clasificación y rotaciones prematuras de plantilla.

Si se considera un escenario conservador en el que el sistema contribuye a evitar un fichaje ineficiente valorado en aproximadamente 20.000\,€ anuales, el retorno de la inversión puede calcularse mediante la fórmula clásica:

\[
ROI = \frac{B - C}{C}
\]

donde $B$ representa el beneficio económico estimado y $C$ el coste de desarrollo.

Sustituyendo valores:

\[
ROI = \frac{20.000 - 16.200}{16.200} \approx 0{,}23
\]

Esto representa un ROI aproximado del 23\% en el primer año de uso.

Este cálculo es deliberadamente prudente y se fundamenta en tres aspectos clave. En primer lugar, el sistema no pretende eliminar completamente el error en la toma de decisiones, sino reducir el margen estimado del 30--45\% asociado al análisis exclusivamente manual. En segundo lugar, el beneficio considerado contempla únicamente el impacto directo de evitar un contrato ineficiente, sin incorporar posibles ahorros en horas de análisis técnico ni la reutilización del sistema en temporadas posteriores. En tercer lugar, el coste de desarrollo es una inversión puntual, mientras que el beneficio puede extenderse en el tiempo mediante su uso recurrente.

Desde esta perspectiva, incluso bajo supuestos conservadores, el sistema VPER presenta viabilidad económica para organizaciones Tier 2, especialmente cuando se considera su potencial de reutilización y mejora continua en ciclos competitivos sucesivos.

\subsection{Valor}

El valor aportado por el sistema VPER se manifiesta en múltiples dimensiones interrelacionadas que trascienden el ámbito puramente técnico.

\begin{description}

    \item[Valor competitivo:] el sistema permite una evaluación objetiva y diferenciada por rol, reduciendo los sesgos tradicionales asociados a métricas superficiales. Esto facilita comparaciones más justas entre jugadores con funciones estratégicas distintas y contribuye a una toma de decisiones más fundamentada.

    \item[Valor organizativo:] el sistema constituye una base tecnológica reutilizable que puede integrarse en procesos internos de scouting, análisis post-partido y planificación estratégica. Al estructurar la evaluación del rendimiento como un modelo formalizado y reproducible, la organización reduce la dependencia exclusiva de valoraciones subjetivas.

    \item[Valor reputacional:] la transparencia en la generación de rankings puede mejorar la credibilidad de decisiones relacionadas con la selección de MVP o fichajes estratégicos. En un entorno altamente mediático como los esports, la percepción de objetividad resulta especialmente relevante.

    \item[Valor tecnológico:] el sistema formaliza la evaluación del rendimiento en esports como un problema de optimización orientado a ranking, integrando múltiples fuentes de datos y manteniendo un enfoque interpretable. Esto sitúa el proyecto no solo como una aplicación práctica, sino como una propuesta metodológica transferible a otros títulos competitivos o disciplinas deportivas digitales.

\end{description}

% =================================================================
% BIBLIOGRAFÍA (Añadir fuentes bibliográficas)
% =================================================================
\newpage
\begin{thebibliography}{9}

    \bibitem{Dex23}
    Gonzalez, M. (2023). 
    \textit{How much do Valorant pros make? Salaries, prize money and earnings explained}. 
    Dexerto Esports News.

    \bibitem{Esp24}
    Brown, T. (2024). 
    \textit{Valorant player salaries: How much do VCT pros earn?}. 
    Esports.net.

    \bibitem{Dot22}
    Scott, R. (2022). 
    \textit{Average salaries in Valorant esports and how contracts work}. 
    Dot Esports.

    \bibitem{Earn25}
    Esports Earnings. (2025). 
    \textit{Top earning Valorant players and prize money statistics}. 
    EsportsEarnings Database.
    \end{thebibliography}

\end{document}
